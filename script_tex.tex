%%%%%%%%%%%%%%%%%%%%%%%%%%%%%%%%%%%%%%%%%%%
%% Descrição: Primeiros passos com o Latex
%% Autor: Paulo Junior
%% 30/08/2023
%%%%%%%%%%%%%%%%%%%%%%%%%%%%%%%%%%%%%%%%%%%

%%%% PREAMBULO

% Classe do documento:

%\documentclass{report} % aceita capítulos
\documentclass{article}

% Pacotes:

\usepackage[brazil]{babel}
\usepackage{graphicx, color} 
\usepackage{geometry}
\geometry{a4paper,headsep=1.0cm,footskip=1cm,
lmargin=2cm,rmargin=2cm,tmargin=2cm, bmargin=2.5cm}
\usepackage{multicol}
\usepackage{ragged2e} % Justificar texto \justfying
\usepackage[normalem]{ulem}
\usepackage{amsmath}
\usepackage{amssymb}
\usepackage[mathscr]{euscript}


% Criando cores novas:

\definecolor{azulescuro}{RGB}{10, 0, 100}

% Apresentação do texto:

\title{Meu primeiro texto em $\LaTeX$.}
%\subtitle{Subtítulo do texto}
\author{Paulo Cerqueira Jr.}
\date{30 de agosto de 2023}

%%%% CORPO DO TEXTO

\begin{document}

\pagenumbering{Alph} % Numeração de páginas



\maketitle % Mostra o titulo do arquivo

\thispagestyle{empty} % remove numeração da página atual.

\tableofcontents


%\begin{center}
%\huge
%Notas de aula de Estatística computacional
%\end{center}
%\begin{center}
%Prof. Paulo Junior
%\end{center}
%\vspace{2cm}


Escrevendo o texto

Aprendendo a inserir capítulos, seções e subseções...

%\chapter{Texto para Capítulo}

\section{Texto dentro de uma Seção}

Exemplo de texto Exemplo de texto Exemplo de texto Exemplo de texto Exemplo de texto Exemplo de texto Exemplo de texto

\subsection{Texto dentro de uma subseção}

Exemplo de texto Exemplo de texto Exemplo de texto Exemplo de texto Exemplo de texto Exemplo de texto Exemplo de texto

\subsubsection{Texto dentro de uma subseção}

Exemplo de texto Exemplo de texto Exemplo de texto Exemplo de texto Exemplo de texto Exemplo de texto Exemplo de texto


\section{Aprendendo a usar multicolunas no texto.}

\begin{multicols}{2} %(aqui xx = 2).
Termine o texto com o comando Termine o texto com o comando
Termine o texto com o comando
Termine o texto com o comando
Termine o texto com o comando
Termine o texto com o comando
Termine o texto com o comando
Termine o texto com o comando
Termine o texto com o comando
Termine o texto com o comando
Termine o texto com o comando
Termine o texto com o comando
Termine o texto com o comando
Termine o texto com o comando
Termine o texto com o comando
Termine o texto com o comando
Termine o texto com o comando
\end{multicols}


\section{Alinhamento de texto}

\begin{center}
Termine o texto com o comando
Termine o texto com o comando
Termine o texto com o comando
Termine o texto com o comando
Termine o texto com o comando
Termine o texto com o comando
Termine o texto com o comando
Termine o texto com o comando    
\end{center}


\begin{flushleft}
Termine o texto com o comando
Termine o texto com o comando
Termine o texto com o comando
Termine o texto com o comando
Termine o texto com o comando
Termine o texto com o comando
Termine o texto com o comando
Termine o texto com o comando
\end{flushleft}


\begin{flushright}
Termine o texto com o comando
Termine o texto com o comando
Termine o texto com o comando
Termine o texto com o comando
Termine o texto com o comando
Termine o texto com o comando
Termine o texto com o comando
Termine o texto com o comando

\end{flushright}


\justifying

Termine o texto com o comando
Termine o texto com o comando
Termine o texto com o comando
Termine o texto com o comando
Termine o texto com o comando
Termine o texto com o comando
Termine o texto com o comando
Termine o texto com o comando



\section{Tamanho de texto}


\tiny
Exemplo de texto

\footnotesize
Exemplo de texto

\Large
Exemplo de texto

\Huge
Exemplo de texto

\normalsize
Exemplo de texto


\section{Estilo de texto}


\emph{Exemplo de texto}

\textit{Exemplo de texto}

\textbf{Exemplo de texto}

\texttt{R}

\textsc{Exemplo de texto}

\uline{Exemplo de texto}

\uuline{Exemplo de texto}

\section{Cores}


\textcolor{red}{Exemplo de texto.}

\textcolor{green}{Exemplo de texto.}

\textcolor{azulescuro}{Exemplo de texto}

\section{Símbolos não matemáticos}

\dag \# 10\% de todas as peças! 


\section{Nota de rodapé}

\hspace{1.5cm} Seja A um evento do espaço amostral definido como S\footnote{Referência para a definição de espaço amostral.}.

\vspace{1cm}

Agora observe que o evento B é tão provável quando o evento A.


\section{Apendendo ambientes}

\subsection{Listas e descrições}

Uma lista pode conter:

\begin{itemize}
    \item Fómulas;
    \begin{itemize}
        \item numeradas;
        \item não numeradas.        
    \end{itemize}
    \item Texto;
    \item Símbolos:
    \begin{itemize}
        \item matemáticos;
        \item não matemáticos.        
    \end{itemize} 
\end{itemize}


Passos para programar em \texttt{R}:

\begin{enumerate}
    \item Ter um pc;
    \item Baixar o \texttt{R} para o pc;
    \item Iniciar o \texttt{R} como uma calculadora;
    \item Aprender funções básicas.
\end{enumerate}

Usando a descrição:

\begin{description}
    \item[\%:] Este síbolo faz referênia a um comentário no código em \LaTeX.
    \item[Paulo:] Nome do professor de Estatística Computacional.
    \item[1] Número um.
\end{description}


Seja A um evento que está contido em S. Então responda:

\begin{description}
    \item[a)] Escreva a união de A e B.
    \item[b)] Escreva o complementar de A.
\end{description}

\subsection{Figuras}

O logo da ufpa é 

\includegraphics{LogoUFPA.png}

\newpage

\scalebox{.5}{\includegraphics{LogoUFPA.png}}


\begin{figure}[t]
    \centering
    \scalebox{.5}{\includegraphics{LogoUFPA.png}}
    \caption{Logo da UFPA.}
    \label{fig:logo_ufpa}
\end{figure}

\subsection{Tabelas}

Aprendendo a criar tabelas...

\begin{tabular}{c|c} \hline
    A  & B \\ \hline
    10 & 20 \\
    5  & 15  \\ \hline 
\end{tabular}


Usando o ambiente \texttt{table}...

\begin{table}[h]
    \centering
    \caption{Título da tabela.}
    \vspace{0.3cm}
    \begin{tabular}{c|c} \hline
        A  & B \\ \hline
        10 & 20 \\
        5  & 15  \\ \hline 
    \end{tabular}
    \label{tab:tabela1}
\end{table}


\begin{table}[h]
    \centering
    \caption{Título da tabela.}
    \vspace{0.3cm}
    \begin{tabular}{ccc} \hline
        A  & B  & Total \\ \cline{1-2}
        10 & 20 &  30   \\
        5  & 15 &  20   \\ \hline 
    \end{tabular}
    \label{tab:tabela1}
\end{table}


Seja X uma va discreta com função de probabilidade dada por

\begin{tabular}{c|ccc} 
X      & 0    & 1    & 2  \\ \hline
P(X=x) & 0.25 &  0.5 & 0.25   \\
\end{tabular}

Responda:

\begin{enumerate}
    \item Calcule P(X=0).
\end{enumerate}


Mais exemplos:

\begin{table}[h]
    \centering
    \caption{Tamanho populacional por ano, estado e região.}
    \vspace{0.3cm}
    \begin{tabular}{cllr}
\hline
{\bf Ano}& {\bf Região}& {\bf Estado} & {\bf Pop*}
\\ \hline \hline
2007 & Norte & Amazonas & 3.222 \\ \cline{3-4}
& & Acre & 665 \\ \cline{2-4}
& Sudeste & São Paulo & 39.828 \\ \cline{3-4}
& & Minas Gerais & 19.274 \\ \hline
2008 & Norte & Amazonas & 3.480 \\ \cline{3-4}
& & Acre & 718 \\ \cline{2-4}
& Sudeste & São Paulo & 44.607 \\ \cline{3-4}
& & Minas Gerais & 21.587 \\ \hline
\end{tabular}
    
    \label{tab:my_label}
\end{table}

*População por mil habitantes.



\section{Equações}

Seja $A$ um evento do espaço amostral $\Omega$, então $P(A)\in [0,1]$.

Seja um evento $A$, tal que,

$$A=\{a: a\in \Omega \}.$$

Tome $X$ como uma variável com função definida da seguinte forma,


 $$f(x)=\mid x \mid - \mid x \mid,\ \mbox{para todo} \ x \in [0,1].$$

\subsection{Ambiente equation}

\begin{equation}
    \label{eq:funcaoJuan}
    f(x)=\mid x \mid - \mid x \mid,\ \mbox{para todo} \ x \in [0,1].
\end{equation}


Seja $X$ uma va em que $X\sim N(\mu, \sigma^2/n)$.

Seja $X$ uma va em que 
$$X\sim N(\mu, \dfrac{\sigma^2}{n}).$$

Seja $X$ uma va em que 
$$X\sim N\left(\mu, \dfrac{\sigma^2}{n}\right).$$

Seja $X$ uma va em que 
$$X\sim N\left[\mu, \dfrac{\sigma^2}{n}\right].$$

Seja $X$ uma va em que 
$$X\sim N\left\{\mu, \dfrac{\sigma^2}{n}\right\}.$$


\subsection{Tipos de letras}


$\mathcal{ABCDE}$


Seja $A$ um conjunto em que  $a\in\mathbb{R}$.




\end{document}
